\cleardoublepage

\chapter{Introducción}
\label{makereference1}

Este proyecto tiene como objetivo predecir, a través de herramientas informáticas, la irradiación solar en un punto teniendo en cuenta factores geográficos y meteorológicos.

La irradiación solar se puede usar como entrada de modelos físicos para la predicción de la energía. De esta manera podremos saber lo que podría generar una placa solar en un momento dado.

Ya existen modelos teóricos para saber cuál va a ser la irradiación solar en un lugar, a partir de la inclinación de los rayos solares en función de la longitud, la latitud, el día del año, la hora del día y suponiendo un cielo en ausencia de nubes. Estos modelos no tienen en cuenta los factores climáticos que tienen un gran peso en la disipación de la radiación.

En la figura \ref{modelo_verano} podemos observar una gráfica en la que aparecen los resultados de distintos modelos teóricos de predicción de la radiación en el verano de 2015. La gráfica representa la radiación media de todo el verano durante las horas del día.

\begin{figure}[htb]
	\begin{center}
		\includegraphics[width=12cm]{figures/verano2015.png}
		\caption{Modelo verano 2015. \label{modelo_verano}} 
	\end{center}
\end{figure}

Aunque se observa en la gráfica que la predicción se acerca a la realidas, la radiación solar no solo se ve afectada por los parámetros de estos modelos. También varía debido a diversos factores de las variaciones locales en la atmósfera como por ejemplo el vapor de agua, las nubes o la contaminación. Este proyecto quiere acercar todavía más esta predicción haciendo uso de algunos de estos factores.

Para llevar a cabo esta predicción en este proyecto se han utilizado como variables atmosféricas la humedad, la temperatura y la propia radiación solar. La manera de resolver este problema han sido algoritmos de ``Machine Learning'' (explicados en el capítulo \ref{makereference5}).

\section{Motivación}
\label{makereference1.1}

Hoy en día es cada vez más importante el uso de las energías renovables y así, utilizar cada vez menos los combustibles fósiles. Esto es debido a que estas no producen gases de efecto invernadero que son los causantes del cambio climático, tampoco producen emisiones contaminantes, son inagotables y generan residuos fáciles de tratar.

También tienen algunos inconvenientes como, por ejemplo, el gran impacto visual que tienen y las grandes cantidades de terreno que se necesitan para poder conseguir una cantidad significante de energía. Además no siempre se obtiene la misma cantidad de energía. Depende, por ejemplo, de la cantidad de sol o de viento en ese momento.

Este último inconveniente es el que hace que las compañías productoras y distribuidoras sean reacias al uso de estas energías.

Debido a la liberalización del sector energético, la compra-venta de la energía se realiza en subastas cada hora. Las compañías, deben conocer su capacidad de producción para poder adecuarla a la demanda de estas subastas. Debido a la incertidumbre de cuánta energía se va a producir, es difícil desarrollar estrategias de compra de energía a otros productores y adecuación de su producción y capacidad de reserva.

En energías renovables como la eólica, los modelos predictivos disponibles actualmente son bastante precisos y su impulso es más fácil que otras como la energía solar la cual es más dificil de predecir.

Se podría decir que todavía no existen modelos realmente buenos para predecir la energía solar. Hay algunas empresas como \href{https://aleasoft.com/es/}{Aleasoft}, que realizan estudios predictivos obteniendo previsiones horarias en tiempo real desde un día hasta diez días, pero sería necesario conocer la producción con un intervalo de una o dos horas.

Para ello, este proyecto intenta predecir a corto plazo la radiación solar y poder aplicar esta en una instalación de granja solar.

\section{Breve descripción del sistema}
\label{makereference1.2}

Este sistema cuenta con cuatro grandes módulos de trabajo para llevar a cabo su función: nodo, servidor de datos, servidor de resultados y visualizador de datos. Ver figura \ref{diagrama-sistema}.

\subsection{Nodo}
\label{makereference1.2.1}
Encargado de recoger los datos necesarios, está formado por distintos sensores que recogen la información meteorológica necesaria y una pequeña placa programable encargada de enviar esta información al servidor de datos.

\subsection{Servidor de datos}
\label{makereference1.2.2}
El servidor de datos, es el encargado de recibir, almacenar y distribuir la información obtenida por el nodo.

\subsection{Servidor de predicción}
\label{makereference1.2.3}
Tiene como función recoger la información almacenada del servidor de datos, procesarla para obtener la predicción y enviarla al visualizador de datos. Puede estar, o no, en la misma máquina que el servidor de datos.

\subsection{Sistema de visualización}
\label{makereference1.2.4}
Muestra los resultados en forma de gráficas para facilitar su análisis.

\begin{figure}[htb]
    \begin{center}
        \includegraphics[width=15cm]{figures/diagrama-sistema.png}
        \caption{Diagrama de los componentes del sistema con sus protocolos de comunicación. \label{diagrama-sistema}}
    \end{center}
\end{figure}
