\cleardoublepage

\chapter{Introduction}
\label{makereference2}

This project aims to predict, through computer tools, the solar radiation at one point taking into account geographic and meteorological factors.

There are already theoretical models to know what the solar irradiation will be in a place, from the inclination of the sun rays in the function of the length, the latitude, the day of the year, the time of day and assuming a sky in the absence of clouds. These models do not take into account the climatic factors that have a great weight in the dissipation of the radiation.

In the picture \ref{modelo_verano} we can observe a graph showing the results of different theoretical models of radiation prediction in the summer of 2015. The graph represents the average radiation of the whole summer during the hours of the day.

Although it is observed in the graph that the prediction approaches the reality, the solar radiation is not only affected by the parameters of these models. It also varies due to various factors of local variations in the atmosphere such as water vapor, clouds or pollution. This project wants to bring this prediction even closer by making use of some of these factors.

To carry out this prediction in this project have been used as atmospheric variables the humidity, temperature and the solar radiation itself. The way to solve this problem have been algorithms of `` Machine Learning '' (explained in chapter \ref{makereference5}).

\section{Motivation}
\label{makereference2.1}

Nowadays, the use of renewable energies is becoming more and more important and, therefore, to use less and less fossil fuels. This is because they do not produce greenhouse gases that are the cause of climate change, they do not produce pollutant emissions, they are inexhaustible and generate waste easy to treat.

They also have some drawbacks, such as the great visual impact they have and the large amounts of land needed to get a significant amount of energy. In addition you do not always get the same amount of energy. It depends, for example, on the amount of sun or wind at that time.

This last drawback is that makes the producing and distributing companies are reluctant to use these energies.

Due to the liberalization of the energy sector, energy buying and selling takes place at hourly auctions. The companies must know their production capacity in order to adapt it to the demand of these auctions. Due to the uncertainty of how much energy is going to be produced, it is difficult to develop strategies to purchase energy from other producers and to adjust their production and reserve capacity.

In renewable energies such as wind, currently available predictive models are quite accurate and their momentum is easier than others such as solar energy which is more difficult to predict.

It could be said that there are still no really good models to predict solar energy. There are some companies like \href{https://aleasoft.com/es/}{Aleasoft}, that perform predictive studies obtaining real-time forecasts from one day to ten days, but it would be necessary to know the production with an interval of one or two hours.

For this, this project tries to predict in the short term the solar radiation and to be able to apply this in a solar farm installation.

\section{Brief description of the system}
\label{makereference2.2}
This system has four large work modules to perform its function: node, data server, results server and data viewer. View picture \ref{diagrama-sistema}.

\subsection{Node}
\label{makereference2.2.1}
In charge of collecting the necessary data, it is formed by different sensors that collect the necessary meteorological information and a small programmable board in charge of sending this information to the data server.

\subsection{Data server}
\label{makereference2.2.2}
The data server is responsible for receiving, storing and distributing the information obtained by the node.

\subsection{Prediction server}
\label{makereference2.2.3}
It has as function to collect the stored information of the data server, to process it to obtain the prediction and to send it to the data visualizer. It may or may not be on the same machine as the data server.

\subsection{Display system}
\label{makereference2.2.4}
Display the results in the form of graphs for easy analysis.