\cleardoublepage

\chapter{Conclusions}
\label{makereference10}

\section{Project status}

The milestones we have successfully developed are:
\begin{itemize}
\item Set up a data collection system. A very simple node to replicate and it could easily work on a network of these. Thanks to the MQTT protocol that allows many publishers since at the time of the training, the identifier of the node with which it was worked was taken into account, the data collected by multiple nodes could be added to the study. An extension of land.

\item Data training system. The software developed for obtaining and processing data for training allows to obtain data sets in a parameterized way. In this way, in a possible continuation of the project, it would be very easy to generate their own data sets for training. In addition, we have explored certain predictive models that, even without much success, would not need to re-study with our parameters.

\item Prediction system. Although an optimal prediction model has not been achieved, the system that receives the collected data has been developed and uses the model chosen to make a prediction. In addition, due to the modularity of our system, it is very easy to implement new prediction models.
\end{itemize}

Each of the components of the system has been developed with a possible future replacement or improvement. Thus, each component is autonomous and easily replaceable by another that respects the communication interface. This allows that the possible deficiencies can be solved without the necessity to alter the rest of components.

This project leaves a margin of improvement over the initial requirements:

\begin{itemize}
\item Predictive model. It has not been possible to obtain a prediction sufficient to be able to take this system to a real environment. One of the possible improvements would be to study other models and parameters.

\item Display system. Due to the lack of time to sample the node, this component could not be polished. The information shown does not appear clearly. However, it does receive and store the necessary data and in the future it would only be necessary to correct the piece of code that transforms raw data into representative charts.
\end{itemize}

\section{Possible extensions}
From the outset, it was possible to form a network of nodes for the collection of samples. It has begun to develop one so as not to expand the complexity of the project.

It is intuited that within the predictive model, it could be interesting to have samples from different points of a territory. Perhaps this could allow to infer to the predictive model changes in the meteorological conditions in a node from those collected by another.

Finally, another possible extension would be to study the prediction model \href{https://www.analyticsvidhya.com/blog/2016/02/time-series-forecasting-codes-python/}{ARIMA}. It makes use of linear regression taking into account past values.